\section{Projects \small{more projects can be found on \href{https://www.github.com/isobarbaric}{\underline{my GitHub Page}}}}

\runsubsection{Teenfaucis API}
\descript{| Python \\ database generation of validated net-publications on COVID-19}
\begin{tightemize}
\item a powerful web-scraper tool that can pin-pointedly harvest factual COVID-19 information
\item empowered with built-in source reliability validation for unadulterated outcomes   
\item ready for deployment under a customized use-case in minutes 
\item \textbf{Tools Used:} requests library, BeautifulSoup library, \href{https://mediabiasfactcheck.com/pro-science/}{MBFC's} categorization 
\end{tightemize}
\sectionsep

% \item created a scraper using Python's BeautifulSoup library to parse through 100+ scientific websites and obtained underlying HTML
% \item parsed through webpages and collected articles that contained COVID-19 related keywords to obtain trusted articles about COVID-19
% \item returned a refined list containing links and titles of relevant articles to the COVID-19 pandemic

\runsubsection{Word Predictor}
\descript{| Java \\ applicability of Tries in auto-complete implementations}
\begin{tightemize}
\item auto-completes words with pool of recommendations for researchers \item scalable prototype to accomodate large data sets for comprehensive applicability under a prefix tree data structure 
\item \textbf{Tools Used:} in-built dynamic data structures, graph traversals, DocumentListener, J-family GUI tools, classes and objects (OOP)
\end{tightemize}
\sectionsep

% \item implemented a Trie data structure to allow efficient storage, access, and manipulation of strings and used an instance of a Trie to add words to it from a dictionary (parsing)
% \item set up a GUI with a DocumentListener object to provide live updates of user inputs to the GUI
% \item displayed possible words from the Trie (using a depth-first search graph-traversal) to emulate auto-complete 

\runsubsection{The Periodic Table: Dynamic Programming}
\descript{| C++ and Python \\ electron config generation using dynamic programming}
\begin{tightemize}
\item proof of concept for the applicability of the paradigm of dynamic programming to the context of periodic chemistry 
\item elemental electron configs follow a pattern that dynamic programming can harness
\item complex ideas made simple using out-of-the-box thinking 
\item \textbf{Tools Used:} fstream library, dynamic data structures, classes and objects (OOP), C header and C++ implementation file distribution
\end{tightemize}
\sectionsep

% \item based on an observed parallel between the Periodic Table and the paradigm of dynamic programming
% \item using the closest noble gas appearing prior to a specific element, electronic configs were generated using the Aufbau principle
% \item aspects inside an atom (shells, subshells, electrons, etc.) were modeled with various classes organized into different C header files 
